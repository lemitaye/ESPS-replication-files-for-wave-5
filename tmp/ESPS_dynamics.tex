\documentclass[11pt]{beamer}
\usepackage[utf8]{inputenc}
\usepackage[T1]{fontenc}
\usepackage{lmodern}
\usetheme{default}
\begin{document}
\author{Lemi Daba}
\title{Comparing adoption of CGIAR related innovations across waves in the ESPS}
%\subtitle{}
%\logo{}
%\institute{}
%\date{}
%\subject{}
%\setbeamercovered{transparent}
%\setbeamertemplate{navigation symbols}{}
\begin{frame}[plain]
	\maketitle
\end{frame}

\begin{frame}
\label{frstpg}
\frametitle{Comparing adoption rates}
\begin{itemize}
\item An increase in adoption of crossbred poultry and large ruminants. The increase is largest for poultry. Also, a notable increase in adoption of improved forages.\hyperlink{animaldynplt}{\beamergotobutton{Figure}}

\item SWC practices adoption has remained virtually the same at hh level, but decreased at EA level. Afforestation, on the other hand, has gone up a bit both at EA and household levels. \hyperlink{swcaffdynplt}{\beamergotobutton{Figure}}

\item CA-zero tillage has remained low at hh level and decreased at EA level but CA-min tillage has gone up. \hyperlink{cadynplt}{\beamergotobutton{Figure}}

\item Adoption of chick-pea Kabuli has decreased from wave 3 to wave 5, with the largest decrease in Amhara and Oromia.  \hyperlink{kabulidynplt}{\beamergotobutton{Figure}}

\item There has been an uptake in agroforestry, both at the EA and hh levels. \hyperlink{agroforestdynplt}{\beamergotobutton{Figure}}

\end{itemize}

\end{frame}



\begin{frame}
\label{scndpg}
\frametitle{Correlates of adoption}
\begin{itemize}
\item Table 14 (add link) still indicates substantial heterogeneity in the types of farmers and communities potentially exposed to the different innovations

\item Large ruminant and poultry crossbreeds are more likely to be adopted by wealthier farmers (as measured by consumption, asset, and off-farm income) and farmers with larger landholdings. \hyperlink{tab14fig1}{\beamergotobutton{Figure}}

\item Adoption of maize germplasm, and drought tolerant maize varieties still remain uncorrelated with most variables. 

\item Adoption of Kabuli chickpea is now not related to farm size (may have something to do with the decreased adoption rate) but seems to show an association with gender (i.e., female managers and female participation in farm activities). \hyperlink{tab14fig2}{\beamergotobutton{Figure}}

\end{itemize}

\end{frame}


\begin{frame}
\label{thrdpg}
\frametitle{Correlates of adoption}
\begin{itemize}

\item River diversion, this time, is more likely to be adopted by smallholder farmers (i.e., negatively related to farm size) but remains uncorrelated with most other covariates

\item SWC \& CA-MT are still positively correlated with farm size and productive asset index but show no correlation with consumption measures of welfare (i.e., annual consumption per capita and being in the bottom 2 quintiles of annual consumption) 

\item Agroforestry (Mango, Papaya, \& Avocado trees) is still not correlated with farm size, negatively correlated with consumption per capita and positively correlated with productive asset index. 
\end{itemize}

\end{frame}



\begin{frame}
\label{frthpg}
\frametitle{Joint adoption rates and synergies}
\begin{itemize}

\item Joint adoption rates of NRM (i.e., SWC \& AWC practices) with most innovations has gone up. \hyperlink{nrmjointplt}{\beamergotobutton{Figure}}

\item The same is true with conservation agriculture (\hyperlink{cajointplt}{\beamergotobutton{Figure}}) and agroforestry (\hyperlink{agrfrstjointplt}{\beamergotobutton{Figure}}). 

\item (maize DNA joint adoptions)

\item Synergies are symmetric and don't show much change from the last wave. \hyperlink{synergydyn}{\beamergotobutton{Figure}}

\end{itemize}

\end{frame}



\begin{frame}
\label{animaldynplt}
\includegraphics[width=\textwidth]{figures/animal_dyn_plt.pdf}
\hyperlink{frstpg}{\beamerreturnbutton{Back}}
\end{frame}


\begin{frame}
\label{cadynplt}
\includegraphics[width=\textwidth]{figures/ca_dyn_plt.pdf}
\hyperlink{frstpg}{\beamerreturnbutton{Back}}
\end{frame}

\begin{frame}
\label{kabulidynplt}
\includegraphics[width=\textwidth]{figures/kabuli_dyn_plt.pdf}
\hyperlink{frstpg}{\beamerreturnbutton{Back}}
\end{frame}

\begin{frame}
\label{swcaffdynplt}
\includegraphics[width=\textwidth]{figures/swc_aff_dyn_plt.pdf}
\hyperlink{frstpg}{\beamerreturnbutton{Back}}
\end{frame}

\begin{frame}
\label{agroforestdynplt}
\includegraphics[width=\textwidth]{figures/agroforest_dyn_plt.pdf}
\hyperlink{frstpg}{\beamerreturnbutton{Back}}
\end{frame}


\begin{frame}
\label{tab14fig1}
\includegraphics[width=\textwidth]{figures/tab14fig1}
\hyperlink{scndpg}{\beamerreturnbutton{Back}}
\end{frame}


\begin{frame}
\label{tab14fig2}
\includegraphics[width=\textwidth]{figures/tab14fig2}
\hyperlink{scndpg}{\beamerreturnbutton{Back}}
\end{frame}


\begin{frame}
\label{nrmjointplt}
\includegraphics[width=\textwidth]{figures/nrm_joint_plt.pdf}
\hyperlink{frthpg}{\beamerreturnbutton{Back}}
\end{frame}

\begin{frame}
\label{cajointplt}
\includegraphics[width=\textwidth]{figures/ca_joint_plt.pdf}
\hyperlink{frthpg}{\beamerreturnbutton{Back}}
\end{frame}


\begin{frame}
\label{agrfrstjointplt}
\includegraphics[width=\textwidth]{figures/agrfrst_joint_plt.pdf}
\hyperlink{frthpg}{\beamerreturnbutton{Back}}
\end{frame}


\begin{frame}
\label{synergydyn}
\includegraphics[width=\textwidth]{figures/synergy_dyn}
\hyperlink{frthpg}{\beamerreturnbutton{Back}}
\end{frame}









































































\end{document}